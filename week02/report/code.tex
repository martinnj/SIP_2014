\section{Task X.X.X Code}
\begin{verbatim}
\end{verbatim}


\section{Task 2.1.3 Code}
\begin{verbatim}
% Solution for part 1.3 of Assignment 2.
% Written by: Martin Jørgensen, tzk173

clear all;

I = imread('pout.tif');

hist = imhist(I);

cumhist = cumsum(hist);

h = figure(213); set(h,'Color','White');
subplot(1,2,1); bar(hist,'FaceColor','black'); axis tight ; grid on;
set(gca,'Xtick',1:16:300); set(gca,'TickDir','out');
title('Histogram for pout.tif','FontSize',14);

subplot(1,2,2); bar(cumhist,'FaceColor','black'); axis tight ; grid on;
set(gca,'Xtick',1:16:300); set(gca,'TickDir','out');
title('Cummulative Histogram for pout.tif','FontSize',14);
\end{verbatim}

\section{Task 2.1.4 Code}
\subsection{FPI function}
\begin{verbatim}
function [ CI ] = fpi( I, CDF )
%fpi Applies the CDF function to all pixels in the image.

CI = arrayfun(@(x) CDF(x), double(I));

end
\end{verbatim}

\subsection{FPI function test}
\begin{verbatim}
% Solution for part 1.4 of Assignment 2.
% Written by: Martin Jørgensen, tzk173

clear all;

I = imread('pout.tif');

% Test the fpi function by using a simple addition function.
Iout = fpi(I,@(x) x+100.4);

% Test some pixels :)
res0 = (Iout(1,1) - double(I(1,1)) == 100.4)
res1 = (Iout(123,200) - double(I(123,200)) == 100.4)
res2 = (Iout(3,200) - double(I(3,200)) == 100.4)
\end{verbatim}