\section{Task 2.1 Histogram-based processing}

\subsection{Task 2.1.1}
The CDF is the integration of the PDF over certain value ranges. As such the PDF
is the probability of a pixel becoming exactly ``some value $x$'', while the CDF
is the probability that a pixel takes a value ``smaller than or equal to $x$''.
Variations in the CDF will thus represent an increased probability for the
distinct value occurring. This behaviour is similar to the histogram and
cumulative histogram in Figure \ref{fig:1_3}.


\subsection{Task 2.1.2}
The PDF of a constant image would be $1$ for the value that is constant in the
image and $0$ for all other. The CDF would be $0$ until the constant value is
reached, and then become $1$ for the reminder of the range.


\subsection{Task 2.1.3}
The code for this task can be found in the appendix. An increase in of the
function means that the intensity value corresponding to the $X$ coordinate of
the bar is present in the picture. The higher the increase the more frequent the
value is in the image. Flat regions represent intensities that are not present
in the image.

\graphicc{1.0}{img/1_3.png}{The histogram, normalized histogram and cumulative
  histogram for the file \textit{pout.tif}.}{fig:1_3}


\subsection{Task 2.1.4}
Code can be found in the appendix. I created the function \texttt{fpi} that
takes an image (or any Matlab array)  and applies a CDF to it.

\graphicc{0.7}{img/1_4.png}{The picture \textit{pout.tif}, before and after the
  CDF is added using \texttt{fpi}.}{fig:1_4}

Figure \ref{fig:1_4} shows how applying an images CDF to the image itself using
the \texttt{gpi} function will look for the image \textit{pout.tif}.

\subsection{Task 2.1.5}
Genrally the CDF is not invertable since several values can produce the same
probability/result, it is not possible in all cases to calculate the value that
caused a certain probability.

The code for the implementation can be found in the appendix. The solution is a
function that given a function $f$ and a probability $l$ will find a value $s$
determined by $\text{min}\{s | f(s) \geq l \}$. This is done by first
acalculating $f(s)$ for all $s \in [0..255]$ then get the indexes for all the
entries that are greater than or equal to $l$, pulling the $s$ values with that
index and then picking the smallest using \texttt{min}.

The test was simply to pick $l = 0.0239$ and the CDF of \textit{pout.tif} as
CDF. The function returns $0.0240$ which is the next entry in the CDF that is
larger than our $l$.


\subsection{Task 2.1.6}
Code is included in the appendix. For this assignment I created the function
\texttt{histmatch} which tales two images as argument, a source image($C_x$ in
the book.) and a target image ($C_z$ in the book.) and returns a copy of the
source image, but with the matched histogram.  The result of the testrun can be
seen in Figure \ref{fig:1_6}.

\graphicc{1.0}{img/1_6.png}{The original, target and combined images with their
  respective histograms.}{fig:1_6}


\subsection{Task 2.1.7}
The code for this task can be found in the appendix.

\graphicc{1.0}{img/1_7.png}{The original image and the results of applying the
  CDF using \texttt{histap} and \texttt{histeq}.}{fig:1_7}

For this task I took some code from \texttt{histmatch} and created
\texttt{histap} which applies a given histogram instead of calculating it
first. As such the function is very similar.


\subsection{Task 2.1.8}
I ran out of time before I could complete this task.

\subsection{Task 2.1.9}
The code for this task can be found in the appendix.
\graphicc{1.0}{img/1_9.png}{The two movies frames before and after the histogram
  alteration.}{fig:1_9}

For this task I created a function called \texttt{midhist} which creates a
cumulative histogram that corresponds to the average of the input histograms. It
is then applied to the frames resulting in what you see in Figure \ref{fig:1_9}.

For applying this technique to an arbitrary number of images the histogram could
be computed like this
\[
    C_z = \frac{1}{N}\sum^N_{n=1} C_n(I_n)
\]
Where $K$ is the total number of pictures and $I_n$ is the currently processing
picture. $C_z$ should then be applied to all the pictures in the sequence.
