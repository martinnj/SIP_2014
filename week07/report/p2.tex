\section{Task 2}
The code for producing Figure \ref{fig:p21} and Figure \ref{fig:p22} can be found in the appendix.

\graphicc{0.8}{img/p21.png}{The results fo different morphological operations on the source image.}{fig:p21}

For illustrative purposes I showed both opening and closing using the builtin \texttt{imclose}/\texttt{imopen} functions,
and  each substep using the \texttt{imerode} and \texttt{imdilate} functions.

Opening will minimize the amount of clutter and small objects in an image, whereas closing will try
to magnify them, essentially allowing many smaller objects to be unified in one big object. As seen
in Figure \ref{fig:p21} the small ``strands'' running from the bigger blop are all eliminated when
opening the image, while they get beefed up when closing it.

When opening an image we first Erode the image, causing all the small details and objects to disappear,
and the bigger objects to ``shrink''. to counter this shrinking we then dilate the image with the same
structure, causing the objects to grow, but since the smallest objects are gone, they cannot regrow.

When closing an image, we do the opposite, we dilate the image causing small objects to grow and merge,
and the dilate it to shrink the objects back down.

In order to create a single blob using either opening or closing the radius $r$ of the diamond structure should be $r=2$.
Figure \ref{fig:p22}shows the results of using different $r$-values with both opening and closing on the original image.

\graphicc{0.7}{img/p22.png}{The result of suing different $r$ values for the structuring element.}{fig:p22}