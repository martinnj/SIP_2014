\section{Fourier Transform – Practice}
\subsection{Question (a)}
\begin{quote}
  \textbf{Use Matlab to calculate the power spectrum of \textit{lena.tif}. Apply
    the function \texttt{fftshift} and interpret the result representation of
    the image.}
\end{quote}

\graphicc{0.6}{img/p2a.png}{The power spectrum of \textit{lena.tif} rendered with the
\texttt{jet} color map.}{fig:p2a}

The code for producing Figure \ref{fig:p2a} can be found in the appendix.

I am not quite sure what the power spectrum represents, but after discussing with my group we agreed
that it is the change in frequency/intensity in an area of the image.

\subsection{Question (b)}
\begin{quote}
  \textbf{Write two programs: 1 that implements convolution as a nested for loop
    of the spatial representation of the kernel and image, 2 that implements
    the same convolution using Fast Fourier Transformation. Compare the two both
    in terms of the result and the computation time for a number of kernel sizes
    and image sizes.}
\end{quote}

The code to produce Figure \ref{fig:p2b} can be found in the appendix along with
the code for the requested functions. Please be aware that the time is
calculated as an average of 10 iterations, which may take some time on slower
computers.

\graphicc{1}{img/p2b.png}{The result of running my convolution methods as well
  as the time for a single run in seconds.}{fig:p2b}

As the figure shows I had some issues making the FFT method give as clear output
as the for loop method. So while it is currently much faster than the for loop
method, the result is also less visible.  In small version of the image it can
be hard to see that the FFT method actually produces an image from the edge
detection kernel, but when it is magnified it's easy to make our the contours.

It is also clear that the FFT method is vastly faster than the loop method that applies the kernel
to every pixel, one by one.

\subsection{Question (c)}
\begin{quote}
  \textbf{Write a program that adds the function $a_0cos(v_0x + w_0y)$ to
    \textit{lena.tif}, and evaluate and describe the power spectrum of the
    result. Design a filter, which removes and such planar waves given $v_0$ and
    $w_0$.}
\end{quote}

The code for producing Figure \ref{fig:p2c} can be found in the appendix along
with the code for adding and attempting to remove the noise. Please note that the code uses a third party command
from Matlab file exchange\footnote{\url{http://www.mathworks.com/matlabcentral/fileexchange/7943-freezecolors---unfreezecolors}}
to let me use different colormaps in the same figure. This command is the work of John Iversen.\footnote{John Iversen, 2005-10 - john\_iversen@post.harvard.edu}

\graphicc{1}{img/p2c.png}{\texttt{lena.tif} with noise, power spectrum and after 2
  different removal attempts.}{fig:p2c}

We can see that the added noise carries over to the power spectrum.
  
It is possible to remove much of the noise using a simple smoothing kernel, that
is a kernel that takes some values from the area around it and mixes it with the
current pixel value. Figure \ref{fig:p2c} is created by applying the kernel
\[
    k = \begin{bmatrix}
            2 & 2 & 2 \\
            2 & 3 & 2 \\
            2 & 2 & 2
        \end{bmatrix}
\]
This kernel is not based in the values $v0$ and $w0$ of course. I was unable to
find a proper way to utilize these values since the kernel is unaware of it's
current position, it will not be able to scale the values correctly (by $x$ and
$y$) anyway.

I tried to construct a function to counteract the noise based on the fact that
we know most of the function that creates the noise. Since the noise addition is
done by this formulae $I_2 = I+a_0cos(v_0x + w_0y)$ and we know $cos(v_0x +
w_0y)$ we can simple subtract that from the pixel value. This of course means
we don't know the factor $a_0$ but it produces reasonable results, even though
they are not as good as the simple smoothing kernel used above.

\subsection{Question (d)}
\begin{quote}
  \textbf{Write the function \texttt{scale}, which implements convolution with a
    isotropic Gaussian kernel, parametrized with the its standard deviation - the scale. Apply it to
    \textit{lena.tif} for a range of scales.}
\end{quote}

The assignment text seems a little vague or ambiguous, so I chose to interpret this as we should
create a function where you give an image and a ``scale''. The function will then set a standard
deviation for you. The code for producing Figure \ref{fig:p2d} and the code for the \texttt{scale}
function can be found in the appendix.

\graphicc{1}{img/p2d.png}{The scale function used on \textit{lena.tif} with different sizes.}{fig:p2d}

The function sets the standard deviation ($\sigma$) based in the scale, it $\sigma$ value is set to
be one third or the size, based on a tip from the assignment last week.

\subsection{Question (e)}
\begin{quote}
  \textbf{Spatial derivatives may be written as the multiplication of a kernel
    in the Fourier Domain.  Derive the exact relation and discuss its
    practicality.}
\end{quote}

\subsection{Question (f)}
\begin{quote}
  \textbf{Implement a function which takes 2 derivative orders and a
    2-dimensional image, and returns the derivative of the image using FFT.}
\end{quote}
