\section{Fourier Transform – Theory}
\subsection{Question (a)}
\begin{quote}
  \textbf{What is the difference between a Fourier series and the Fourier
    Transform?}
\end{quote}

%\todo[inline]{This is quoted from page 115, in the block with label
%  (5). Rephrase or expand.}

\noindent Page 115 from the book gives the following definitions:
\begin{description}
\item[Fourier Series] breaks down a periodic signal into harmonic functions of
  discrete frequencies.
\item[Fourier Transform] breaks down a non-periodic signal into harmonic
  functions of continuously varying frequencies.
\end{description}

The series best used for a signal that have obvious periods and repeats while
the transform can be applied to any signal, by assuming that the signal period
is going towards $\infty$.

\subsection{Question (b)}
\begin{quote}
  \textbf{Prove that the continuous Fourier transform of a real and even
    function is real and even.}
\end{quote}

% Logical motivation:
I am not great with formalism, but since the point of Fourier transforms is to
represent a function/signal it only makes sense that if the input is real and
even, the output needs to be as well. Below is some argumentation that should
further strengthen the notion that this is indeed the case.
%, or the method would be very poor at representing signals.

% Der står noget på side 119 mellem 5.5 og 5.6 (se 5.6 hvor c_k og c_-k bliver
% udregnet som imaginære tal der går ud med hinanden.)

% Exponential function: e^{a+bi} = e^a (cos b + i sin b) (a in our case is
% always 0, so there will be no e^a in the right side.)

% For some value n the we integrate for both the positive and negative version
% since the integration is from -infty to infty. We thus have:

% below e^0 can be replaced by 1. all of the lifted part of e can be negated
% because a negative n will propegate its sign.

% e^{i\frac{2 pi n}{lambda}} = 1(cos(\frac{2 pi n}{lambda}) + i sin(\frac{2 pi n}{lambda}))
% e^{-i\frac{2 pi n}{lambda}} = 1(cos(\frac{2- pi n}{lambda}) + i sin(\frac{-2 pi n}{lambda}))

Since we integrate from $-\infty$ to $\infty$, we will evaluate for both the
positive and negative value for any $n$.  The only imaginary part of the
transform comes from the exponential function, so lets take a look at that
\[
    e^{a+bi} = e^a (cos(b) + i\cdot sin(b))
\]
In our case $a = 0$ so the exponential function so we have $1(cos(b) + i\cdot sin(b))$
instead. so for both $n$ and $-n$ we get:
\begin{align*}
  e^{i\frac{2 \pi n}{\lambda}} &= 1(cos(\frac{2 \pi n}{\lambda}) + i\cdot sin(\frac{2 \pi n}{\lambda}))\\
  e^{-i\frac{2 \pi n}{\lambda}} &= 1(cos(\frac{2 \pi n}{\lambda}) + -i\cdot sin(\frac{2 \pi n}{\lambda}))
\end{align*}

When we in the end integrate over both of these, the imaginary and negative
imaginary components will ``cancel each other out'' and this ensures that the
resulting transform is real.

\subsection{Question (c)}
\begin{quote}
  \textbf{Derive the continuous Fourier transform of $\delta(x - d) + \delta(x +
    d)$ for some constant $d$}
\end{quote}

Assuming the $\delta$ is the Dirac Impulse Function (which always integrate to
$1$), it does not matter if you add anything to the "x" it just shifts along the
$x$-axis. Since there is 2 delta functions added together, the resulting $F(k)$ is
$2$. (See table 5.2 on page 123), this is just for a single delta function
though.

\subsection{Question (d)}
\begin{quote}
  \textbf{Consider the box function
  \[
    b_a(x) = \begin{cases}
      1/a  & \text{if } |x| \leq \frac{a}{2} \\
      0    & \text{otherwise.}
      \end{cases}
  \]
  show that}
  \begin{itemize*}
  \item[i)] \textbf{$\int_{-\infty}^\infty b_a(x) dx = 1$}\\

    Since integration is the area under the graph, and we have a box function,
    we can rewrite the integration as a simple $a = l \times w$ formulae. This
    can then be solved to be 1.
      \begin{align*}
        1 &= \frac{1}{a}2\frac{a}{2} \\
          &= \frac{2}{a} \frac{a}{2} \\
          &= 1
      \end{align*}
      %\todo[inline]{Do a better job at explaining why we can use the simple
      %  formulae rather than the integration.}

    \item[ii)] \textbf{the continuous Fourier transform of b using (5.10) is
        $B(k) = \frac{1}{ak\pi} sin \frac{ak}{2}$. Rewrite $B(k)$ using $sinc(x)
        = \frac{sin(x)}{x}$.}\\
    % The rectangle function can be used maybe?
    \begin{align*}
      B(k) &= \frac{1}{ak\pi}sin\left( \frac{ak}{2}\right) \\
           &= \frac{1}{\pi} \frac{1}{ak} sin \left( ak\frac{1}{2} \right) \\
           &= \frac{1}{2\pi} \frac{1}{ak\frac{1}{2}} sin \left( ak\frac{1}{2} \right) \\
           &= \frac{1}{2\pi} \frac{sin\left( ak\frac{1}{2} \right)}{ak\frac{1}{2}} \\
           &= \frac{1}{2\pi} sinc \frac{ak}{2}
    \end{align*}



    \item[iii)] \textbf{$\text{lim}_{a \rightarrow 0}B(k) = \frac{1}{2\pi}$
        (Hint: $\text{lim}_{x \rightarrow 0}\frac{sin(x)}{x} = 1$). Does this
        prove an entry in Table 5.2?}\\

        By the tip we have $\text{lim}_{x \rightarrow 0} \frac{sin(x)}{x} = 1$, that means that if $a \rightarrow 0$ for
        $\frac{1}{2\pi}sinc \left( \frac{ak}{2} \right)$ we have $\text{lim}_{a \rightarrow 0} \frac{1}{2\pi}sinc \left( \frac{ak}{2} \right) =
        \frac{1}{2\pi}1$ which equals $\text{lim}_{a \rightarrow 0} B(k) = \frac{1}{2\pi}$.


    \item[iv)] \textbf{The filter $b$ has compact support in space (only use a
        small set of neightbouring pixels in $x$). Is the same true in the
        frequency domain, $k$? Explain your answer.}\\
        
    Since the transformation into the frequency domain is reversible, there must be a correspondence
    between the two. So if $b$ has compact support in normal space, it will also have it in the
    frequency space $k$.
  \end{itemize*}
\end{quote}