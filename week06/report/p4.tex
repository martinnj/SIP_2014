\section{Task 4: Affine and projective alignments}

\subsection{Part 1}

The main appeal is the ability to use one single matrix multiplication to do all
the transformations needed. It also allows us to append transformations as a
chain of matrix multiplications.  A single transformation would be
\begin{align}
    \begin{bmatrix} x' \\ y' \\ 1 \end{bmatrix} &=
      M \begin{bmatrix} x \\ y \\ 1 \end{bmatrix} \label{eq:step1}
\end{align}
Since the result of the multiplication in Equation \ref{eq:step1} is also a
3-element vector, we can simply chain multiplications on like so
\begin{align}
    \begin{bmatrix} x' \\ y' \\ 1 \end{bmatrix} &= 
      M_n, M_{n-1}, ..., M_0 \begin{bmatrix} x \\ y \\ 1 \end{bmatrix} \label{eq:step2}
\end{align}


\subsection{Part 2}

The code for producing Figure \ref{fig:p42} can be found in the appendix.

\graphicc{0.8}{img/p42.png}{Left: $X$ and $Y$ drawn in the same plot. Right: Same plot, but with a Procrustes aligned X overlaid.}{fig:p42}

It will not be possible to get an exact match for $X$ mapped to $Y$ since the
edges in $X$ are pairwise parallel, whereas only 2 edges in $Y$ are
parallel. Neither translation, rotation or scaling, the angles between edges in the figure.
Figure \ref{fig:p42} shows the closest mapping Procrustes analysis can create.

\subsection{Part 3}
The code for producing Figure \ref{fig:p43} can be found in the appendix.
\graphicc{0.8}{img/p43.png}{Left: $X$ and $Y$ drawn in the same plot. Right: Same plot, but with a Affine transformed X overlaid.}{fig:p43}

Figure \ref{fig:p43} shows the result of the affine transformation, we can see that because we're also
able to shear the figure, we get a slightly closer match. We are still not able to break the parallel relationship of edges though.


\subsection{Part 4}
The code for producing Figure \ref{fig:p44} can be found in the appendix.
\graphicc{0.8}{img/p44.png}{Left: $X$ and $Y$ drawn in the same plot. Right: Same plot, but with a Projective transformed X overlaid.}{fig:p44}

Figure \ref{fig:p44} is the result of using Projective transformation on X. The ability to manipulate
$X$ in 3D space allows us to break the parallel relationship of edges and make a near perfect match.

According to Matlab there is some small differences, but they seem so small that it might be due to
lost floating point precision in the processing. The transformed $X$ and $Y$ are to all intents and
purposes matching.