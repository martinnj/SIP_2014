\section{Task 2: Procrustes transformations}

\subsection{Part 1}

%There is 4 free variables for the Procrustes transformation, see page 176 for details.
%Figure 7.12 is the transformation matrix and contains them all.

%The value for N is mentioned in section 7.7.

%Need 1 for translation and two for scale.

For translation only, the minimum $N$ is 1, since we can do translation from one point to another, by subtracting them.
The same goes for scaling, except it is simply multiplying the single point until the MSE stops falling.


\subsection{Part 2}

The code for producing the values in Equation \ref{eq:22} and Figure
\ref{fig:p22} can be found in the appendix.

Filling the values printed from \texttt{p22.m} into Equation (7.13) from p. 176
into the matrices we get the result show in Equation \ref{eq:22}.

\begin{align}
  \text{\textbf{X}} &= \begin{bmatrix}
    1 & 0 & -80.4378 \\
    0 & 1 & 6.2076   \\
    0 & 0 & 1
  \end{bmatrix},
  \text{\textbf{R}} = \begin{bmatrix}
    0.9864 & -0.1647 & 0 \\
    0.1647 & 0.9863  & 0 \\
    0 & 0 & 1
  \end{bmatrix},
  \text{\textbf{S}} = \begin{bmatrix}
    1.0893 & 0 & 0 \\
    0 & 1.0893 & 0 \\
    0 & 0 & 1
  \end{bmatrix}
  \label{eq:22}
\end{align}

\graphicc{1}{img/p22.png}{The two images and the result of performing a
  procrustes transformation and overlaying the result with the
  original.}{fig:p22}

  In Figure \ref{fig:p22} we see that the features of the images overlay very well meaning the
  MRE for the alignment will be very small.