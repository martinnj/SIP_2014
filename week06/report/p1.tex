\section{Task 1: Experiments on translations}

\subsection{Part 1}
The general recipe for creating a translation transformation matrix is
\begin{align}
    M_\text{trans} &= \begin{bmatrix}
                        1 & 0 & x_d \\
                        0 & 1 & y_d \\
                        0 & 0 & 1 
                      \end{bmatrix}
                      \label{eq:mtrans} \\
    \begin{bmatrix}
        g_x \\ g_y \\ \_
    \end{bmatrix}    &= M_\text{trans} \times \begin{bmatrix}
                                      f_x \\ f_y \\ 1
                                    \end{bmatrix} \label{eq:mapply}
%\caption{General transformation matrix and formula for applying it.}
\end{align}
where $x_d$ and $y_d$ are the offsets we wish to impose on the $x$ and $y$-axis respectively and
$f_x$ and $f_y$ are coordinates in the original image.

Using Equation \ref{eq:mtrans} the matrix for translating 1 pixel to the right is 
\begin{align*}
       M &= \begin{bmatrix}
              1 & 0 & x_d \\
              0 & 1 & 0   \\
              0 & 0 & 1   \\
            \end{bmatrix}.
\end{align*}

If we wish to create a kernel/filter for translating $1px$ towards the right, we use the following matrix
\begin{align*}
    M_{filter} &= \begin{bmatrix}
                    0 & 0 & 0 \\
                    1 & 0 & 0 \\
                    0 & 0 & 0
                  \end{bmatrix}
\end{align*}
The code for creating this matrix is found in the appendix.

When convolved with an image this filter will make all pixels assume the value of their left
neighbour, effectively translation the entire image $1px$ to the right.

\subsection{Part 2}

Using the method described in Equation \ref{eq:mtrans} and \ref{eq:mapply} I
created the function \texttt{MyTranslate} which takes an image and two integers
as input, the integers are the offset to translate with. The code for the
function and for creating Figure \ref{fig:p12} as well as the code for the
function can be found in the appendix.

\graphicc{0.8}{img/p12.png}{The image before and after translation.}{fig:p12}

The handling of boundary conditions in my implementation is simple, all of the output
image is initialized to 0. This means that pixels that don't get assigned a
value will just stay zero, coordinates from the old image that translates
outside the bounds of the output (which have the same size as the input), are
simply ignored.

\subsection{Part 3}

Translation in the Frequency domain is given by
\begin{align}
    f(x-\Delta x, y-\Delta y) &= F(u,v) e^{-i2\pi (\frac{u\Delta x}{N} + \frac{v\Delta y}{M})}
\end{align}

Where $M$ is the number of rows in the image, and $N$ is the number of columns.

\graphicc{0.8}{img/p13.png}{The image before and after translation.}{fig:p13}

The code for producing Figure \ref{fig:p13} can be found in the appendix along
with the code for my translation function called \texttt{MyFTranslate}.

\subsection{Part 4}

The code for producing Figure \ref{fig:p14} can be found in the appendix.

\graphicc{0.8}{img/p14.png}{The image before and after translation.}{fig:p14}

We can see that the translation does not go smoothly, in the lena image we see
that the image gets significantly brighter when it gets translated, and in the
point image we can see why, the bright spots gets ``smeared'' out in the
directions the image gets translated.
