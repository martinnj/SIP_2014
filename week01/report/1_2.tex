\section{Task 1.2}
To solve this I generated a $60\times 60$ image in the same way as Task 1.1, and
displayed it using the three different methods \texttt{imshow}, \texttt{image}
and \texttt{imagesc}. They are descriped in the list below.

\graphicc{0.95}{img/1_2.png}{The same image displayed using the different
  methods.}{fig:1_2}

\begin{description}
\item[\texttt{imshow}] Can take either a string with a filename/path, an image
  handle or a matrix as input and displays it as an image. It has a number of
  default attributes, such as disabling axis labels and ticks, and forcing
  square pixels by making the x and y axis have the same distance between ticks.
  It will also try to guess whether an image/matrix us in grayscale, RGB or
  binary.
\item[\texttt{image}] Takes a matrix as input which it will interpret as an
  image. The resulting image object will attempt to stretch to fill whatever
  window it is nested in, resulting in rectangular pixels. It will keep the tick
  marks along the images. Will use the active colormap instead of trying to
  guess what the data means.

\item[\texttt{imagesc}] Have the same behaviours as image, but will scale the
  image data to make use of the full colormap.
\end{description}

The source code is in the appendix.