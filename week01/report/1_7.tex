\section{Task 1.7}
\graphicc{0.95}{img/1_7.png}{The results fo the different methods and
  operations.}{fig:1_7}

The two first pictures in Figure \ref{fig:1_7} is the result of two methods of
adding images. Since the image is bound by the 8bit intensity limit, they
produce the same result. The add the intensities of the images together, bound
it at 255 and displays it. When letting \texttt{imadd} (upper right image) give
16 bit output, the result is vastly different. The increased number of different
shades means that values that become higher than 255 does not jost go white. The
images are showed with \texttt{imagesc} in order to increase the visibility of
the number of intensities.

For the last two images, just like the first, there is no
difference. Intensities cannot go below 0, and as such they just go black. It is
not possible to specify 16bit output for \texttt{imsubtract}, but it would not
help, since we cannot display a negative intensity.
