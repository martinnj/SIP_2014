\section{Task 1.5}
The code used to generate the figures can be found in the appendix.
\graphicc{0.8}{img/1_5_downscale.png}{The Cookie monster image sized down using
  three different scales. The last one also ruins the aspect ratio of the
  image.}{fig:1_5_down}

In the upper right image, ther eis clear aliasing as the resolution falls and
``smooth'' edges can no longer be drawn. It is still possible to make out all
the details, evne though the curls in Cookie Monsters fur and the
ligh-reflections on the green lego brick is harder to make out.  In the lower
left image on Figure \ref{fig:1_5_down} several details are gone, the pupils in
Cookie Monsters eyes as well as the reflections on the lego block are almost
impossible to make out. The text on the jar is unreadable. The lower right image
in the same figure does not loose the details, but because the aspext-ratio is
skewed, some details like the text on the jar is a bit hard to make out.

\graphicc{0.8}{img/1_5_upscale.png}{Results when scaling up a low resolution
  image, in this case the original is the cookie monster picture scaled
  down.}{fig:1_5_up}

The upper right image in Figure \ref{fig:1_5_up} is created by giving all the
new pixels the same value as the pixel that is nearest their position in the
original image. This essentially ``stretches'' all the pixels and gives a very
jagged look. When comparing it to the original they're identical since they're
displayed in the same size. The two lower figures are taken by giving new pixels
a value based on the average value in their neighborhood in the original
image. ($2\times 2$ and $4\times 4$ neighborhoods respectively.) The edges gets
a lot softer and less jagged, but the image gets a very fussy quuality to it, as
if the camera was out of focus.
