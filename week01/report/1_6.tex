\section{Task 1.6}
The code for this task is shown in the appendix.

\graphicc{0.4}{img/1_6.png}{Measurements (in pixels) of the railway sleepers.}{fig:1_6}

Figure \ref{fig:1_6} shows the measurements of the sleepers on the railway. 2 parallel lines were
added at the end of the sleepers in order to facilitate easier measurements of the sleepers that are
further into the background, since these were blurry.

The first measures sleepers is $165,5$ pixels across. The second is $77.5$ px and the third is
$35.75$ px. It is impossible to recreate the $(X,Y,Z)$ positions of the sleepers without any more information.
We lack both focal length, and some information about positions in the scene. We can use the rate at
which the sleepers get shorter, i.e. how fast the lines that go to the vanishing point converges, to
make some assumptions about how ``wide'' things in the picture are along the $x$-axis, but apart from
that there is little we can do.

Example, I measured the distance between the two ``parallel'' lines to be $4.5$ px, at the same
distance as the first white house, the wall of the white house is roughly $4.5$ px tall and $25.38$ px wide.
Knowing that for every $4.5$ pixel we have roughly $2.5$ meters, the wall must be around $2.5$ meters
tall and $14.1$ meters long. ($\frac{25.38px}{4.5px}2.5m = 14.1m$). This assumes that the wall is
somewhat parallel to the camera.