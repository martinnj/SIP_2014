\section{Task 1.1 Code}
\begin{verbatim}
% Solution for part 1.1 of Assignment 1.
% Written by: Martin Jørgensen, tzk173

clear all;


% Generate figure and image.
figure(111);
pixels = randi([0,1],20,20);
h = imshow(pixels); axis on;

% Fix the looks.
set(gca,'XTick',0:20); xlabel('x');
set(gca,'YTick',0:20); ylabel('y');

% Get input and add the coordinate to the title, then color the
% pixel and update the figure data.
[x,y] = ginput(1);
title(strcat('(' , num2str(round(x)) , ',' , num2str(round(y)) ,')'));
pixels(round(y),round(x)) = 0; set(h,'CData',pixels);
\end{verbatim}

\section{Task 1.2 Code}
\begin{verbatim}
% Solution for part 1.2 of Assignment 1.
% Written by: Martin Jørgensen, tzk173

clear all;

% Create figure and image data.
h = figure(121);
pixels = randi([0,1],60,60);

% Create sub figure for imshow.
subplot (1,3,1)
imshow(pixels)
title('imshow')

% Create sub figure for image.
subplot (1,3,2)
image(pixels)
title('image')

% Create sub figure for imagesc.
subplot (1,3,3)
imagesc(pixels)
title('imagesc')
\end{verbatim}

\subsection{Task 1.3 Code}
\begin{verbatim}
% Solution for part 1.3 of Assignment 1.
% Written by: Martin Jørgensen, tzk173

clear all;

% Read the selected test image.
I = imread('cell.tif');

% Create parent figure.
h = figure(131);

% Go through the bits from least significant to most.
for i=1:8
    subplot(2,4,9-i); % Ordered most-significant leftmost.
    colormap(gray);   % Force colormap.
    imagesc(arrayfun(@(x) bitget(x,i),I)); % apply bitget to each pixel.
    axis image; title(strcat('Bit:', num2str(i-1)));
end
\end{verbatim}

\subsection{Task 1.4 Code}
\begin{verbatim}
% Solution for part 1.4 of Assignment 1.
% Written by: Martin Jørgensen, tzk173

clear all;

% Read the selected test image.
I = imread('monster.jpg');

% Convert it.
[H,S,V] = rgb2hsv(I);

figure; % Create a new figure
subplot(2,2,1); imshow(I); title('Original');  axis image;
subplot(2,2,2); imshow(H); title('Hue'); axis image;
subplot(2,2,3); imshow(S); title('Saturation'); axis image;
subplot(2,2,4); imshow(V); title('Value'); axis image;
\end{verbatim}

\subsection{Task 1.5 Code}
\begin{verbatim}
% Solution for part 1.5 of Assignment 1.
% Written by: Martin Jørgensen, tzk173

clear all;

% Read the selected test image.
I = imread('monster.jpg');

% Make Throw away resolution.
I1 = imresize(I,0.09);
I2 = imresize(I,0.05);
I3 = imresize(I,[100 600]);

figure(151); % Create a new figure.
subplot(2,2,1); imshow(I); title('Original');  axis image on;
subplot(2,2,2); imshow(I1); title('(I,0.9)'); axis image on;
subplot(2,2,3); imshow(I2); title('(I,0.5)'); axis image on;
subplot(2,2,4); imshow(I3); title('(I,[100 600])'); axis image on;


% Upscaling experiments.

% Warmup the image cache to get better times.
for i=1:50
    I4 = imresize(I1,2,'nearest');
end

% Get the proper timings.
tic
I4 = imresize(I1,2,'nearest');
t1 = toc;

tic
I5 = imresize(I1,2,'bilinear');
t2 = toc;

tic
I6 = imresize(I1,2,'bicubic');
t3 = toc;


figure(152);
subplot(2,2,1); imshow(I1); title('Original');  axis image on;
subplot(2,2,2); imshow(I4); axis image on;
title(strcat('Interpolator: Nearest, Runtime in seconds: ',num2str(t1)));
subplot(2,2,3); imshow(I5); axis image on;
title(strcat('Interpolator: Bilinear, Runtime in seconds: ',num2str(t2)));
subplot(2,2,4); imshow(I6); axis image on;
title(strcat('Interpolator: Bicubic, Runtime in seconds: ',num2str(t3)));
\end{verbatim}

\subsection{Task 1.6 Code}
\begin{verbatim}
% Solution for part 1.6 of Assignment 1.
% Written by: Martin Jørgensen, tzk173

clear all;

% Read the selected image.
I = imread('railway.png');

% Sleeper length is 2.5m
% Do proper math in report.
imtool(I);
\end{verbatim}

\subsection{Task 1.7 Code}
\begin{verbatim}
% Solution for part 1.7 of Assignment 1.
% Written by: Martin Jørgensen, tzk173

clear all;

% Read the selected image.
I1 = imread('rice.png');
I2 = imread('cameraman.tif');

% Add the images.
I3 = I1 + I2;
I4 = imadd(I1,I2);
I5 = imadd(I1,I2,'uint16');

% Subtract them.
I6 = I1 - I2;
I7 = imsubtract(I1,I2);

% Display them.
figure(171);
colormap(gray);

subplot(2,3,1);
imagesc(I3); axis image;
title('Matlab + operator.');

subplot(2,3,2);
imagesc(I4); axis image;
title('Matlab imadd() function.');

subplot(2,3,3);
imagesc(I5); axis image;
title('Matlab imadd(X,Y,uint16) function.');

subplot(2,3,4);
imagesc(I6); axis image;
title('Matlab - operator.');

subplot(2,3,5);
imagesc(I7); axis image;
title('Matlab imsubtract() function.');
\end{verbatim}

\subsection{Task 1.8 Code}
\begin{verbatim}
% Solution for part 1.8 of Assignment 1.
% Written by: Martin Jørgensen, tzk173

clear all;

% Read the selected images.
imgs = cell(1,10);
for i=1:9
    imgs{i} = imread(strcat('AT3_1m4_0',num2str(i),'.tif'));
end
imgs{10} = imread(strcat('AT3_1m4_10.tif'));

figure(181);
colormap(gray);
for i=1:9
    subplot(2,5,i);
    imagesc(imabsdiff(imgs{i},imgs{i+1})); axis image;
    title(strcat('Diff: ',num2str(i), '-', num2str(i+1)));
end
\end{verbatim}

\subsection{Task 1.9 Code}
\subsubsection{Blending function}
\begin{verbatim}
function [ C ] = imgBlend( A, B, wA, wB )
  %imgBlend Performs a weighted blend of the images A and B.

  %C = A.*wA + B.*wB;
  C = imadd(immultiply(A,wA),immultiply(B,wB),'uint16');
end
\end{verbatim}

\subsubsection{Experiment/Use}
\todo[inline]{Insert when done.}
\begin{verbatim}
% Solution for part 1.9 of Assignment 1.
% Written by: Martin Jørgensen, tzk173

clear all;

% Read the selected images.
imgs = cell(1,10);
for i=1:9
    imgs{i} = imread(strcat('AT3_1m4_0',num2str(i),'.tif'));
end
imgs{10} = imread(strcat('AT3_1m4_10.tif'));

res = imgBlend(imgs{1},imgs{2},2.5,4);
for i=2:9
    res = imgBlend(imgs{i},imgs{i+1},2.5,4);
end

% Show result.
figure(191);
colormap(gray);
imagesc(res);
\end{verbatim}