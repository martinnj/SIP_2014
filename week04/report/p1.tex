\section{Task 1}
The code for creating Figure \ref{fig:p1} can be found in the appendix.

\graphicc{0.8}{img/p1.png}{The different kernels produced for the task.}{fig:p1}

Figure \ref{fig:p1} shows the four kernels used for this task. The top left is the kernel produced
with the deviation value of $\sigma = 10$, and the upper right us the kernel for deviation value
$\tau = 5$. The lower left figure us the kernel produced when the deviation value is set to
$\sqrt{\sigma^2 + \tau^2}$, while the lower right shows the resulting kernel from convolving the
two upper kernels.
There is a number of remarks to be made here. In order to eliminate errors from ``border cases'',
the size of the kernels was set to be $200\times 200$ so that all the kernels would have 0 weight
values around the edges. The kernels are then cropped so we can see the most interesting part, the
centres.

The two last lower kernels look very similar but it is worth noticing that the kernel produced by
convolution have tiny ``spikes/rays'' emanating from the centre along the x- and y-axis. This
becomes almost invisible when the images are zoomed back out and will only cause minor differences
when applied to pictures.

I made several experiments with different kernel sizes and ratios between $\sigma$ and $\tau$ and
all yielded this same result. The pictures from the experiments are omitted to preserve space. But
can easily be recreated by changing the values in the code in Figure \ref{code:p1} in the appendix.